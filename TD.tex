% Options for packages loaded elsewhere
\PassOptionsToPackage{unicode}{hyperref}
\PassOptionsToPackage{hyphens}{url}
%
\documentclass[
]{article}
\usepackage{lmodern}
\usepackage{amssymb,amsmath}
\usepackage{ifxetex,ifluatex}
\ifnum 0\ifxetex 1\fi\ifluatex 1\fi=0 % if pdftex
  \usepackage[T1]{fontenc}
  \usepackage[utf8]{inputenc}
  \usepackage{textcomp} % provide euro and other symbols
\else % if luatex or xetex
  \usepackage{unicode-math}
  \defaultfontfeatures{Scale=MatchLowercase}
  \defaultfontfeatures[\rmfamily]{Ligatures=TeX,Scale=1}
\fi
% Use upquote if available, for straight quotes in verbatim environments
\IfFileExists{upquote.sty}{\usepackage{upquote}}{}
\IfFileExists{microtype.sty}{% use microtype if available
  \usepackage[]{microtype}
  \UseMicrotypeSet[protrusion]{basicmath} % disable protrusion for tt fonts
}{}
\makeatletter
\@ifundefined{KOMAClassName}{% if non-KOMA class
  \IfFileExists{parskip.sty}{%
    \usepackage{parskip}
  }{% else
    \setlength{\parindent}{0pt}
    \setlength{\parskip}{6pt plus 2pt minus 1pt}}
}{% if KOMA class
  \KOMAoptions{parskip=half}}
\makeatother
\usepackage{xcolor}
\IfFileExists{xurl.sty}{\usepackage{xurl}}{} % add URL line breaks if available
\IfFileExists{bookmark.sty}{\usepackage{bookmark}}{\usepackage{hyperref}}
\hypersetup{
  hidelinks,
  pdfcreator={LaTeX via pandoc}}
\urlstyle{same} % disable monospaced font for URLs
\usepackage{color}
\usepackage{fancyvrb}
\newcommand{\VerbBar}{|}
\newcommand{\VERB}{\Verb[commandchars=\\\{\}]}
\DefineVerbatimEnvironment{Highlighting}{Verbatim}{commandchars=\\\{\}}
% Add ',fontsize=\small' for more characters per line
\newenvironment{Shaded}{}{}
\newcommand{\AlertTok}[1]{\textcolor[rgb]{1.00,0.00,0.00}{\textbf{#1}}}
\newcommand{\AnnotationTok}[1]{\textcolor[rgb]{0.38,0.63,0.69}{\textbf{\textit{#1}}}}
\newcommand{\AttributeTok}[1]{\textcolor[rgb]{0.49,0.56,0.16}{#1}}
\newcommand{\BaseNTok}[1]{\textcolor[rgb]{0.25,0.63,0.44}{#1}}
\newcommand{\BuiltInTok}[1]{#1}
\newcommand{\CharTok}[1]{\textcolor[rgb]{0.25,0.44,0.63}{#1}}
\newcommand{\CommentTok}[1]{\textcolor[rgb]{0.38,0.63,0.69}{\textit{#1}}}
\newcommand{\CommentVarTok}[1]{\textcolor[rgb]{0.38,0.63,0.69}{\textbf{\textit{#1}}}}
\newcommand{\ConstantTok}[1]{\textcolor[rgb]{0.53,0.00,0.00}{#1}}
\newcommand{\ControlFlowTok}[1]{\textcolor[rgb]{0.00,0.44,0.13}{\textbf{#1}}}
\newcommand{\DataTypeTok}[1]{\textcolor[rgb]{0.56,0.13,0.00}{#1}}
\newcommand{\DecValTok}[1]{\textcolor[rgb]{0.25,0.63,0.44}{#1}}
\newcommand{\DocumentationTok}[1]{\textcolor[rgb]{0.73,0.13,0.13}{\textit{#1}}}
\newcommand{\ErrorTok}[1]{\textcolor[rgb]{1.00,0.00,0.00}{\textbf{#1}}}
\newcommand{\ExtensionTok}[1]{#1}
\newcommand{\FloatTok}[1]{\textcolor[rgb]{0.25,0.63,0.44}{#1}}
\newcommand{\FunctionTok}[1]{\textcolor[rgb]{0.02,0.16,0.49}{#1}}
\newcommand{\ImportTok}[1]{#1}
\newcommand{\InformationTok}[1]{\textcolor[rgb]{0.38,0.63,0.69}{\textbf{\textit{#1}}}}
\newcommand{\KeywordTok}[1]{\textcolor[rgb]{0.00,0.44,0.13}{\textbf{#1}}}
\newcommand{\NormalTok}[1]{#1}
\newcommand{\OperatorTok}[1]{\textcolor[rgb]{0.40,0.40,0.40}{#1}}
\newcommand{\OtherTok}[1]{\textcolor[rgb]{0.00,0.44,0.13}{#1}}
\newcommand{\PreprocessorTok}[1]{\textcolor[rgb]{0.74,0.48,0.00}{#1}}
\newcommand{\RegionMarkerTok}[1]{#1}
\newcommand{\SpecialCharTok}[1]{\textcolor[rgb]{0.25,0.44,0.63}{#1}}
\newcommand{\SpecialStringTok}[1]{\textcolor[rgb]{0.73,0.40,0.53}{#1}}
\newcommand{\StringTok}[1]{\textcolor[rgb]{0.25,0.44,0.63}{#1}}
\newcommand{\VariableTok}[1]{\textcolor[rgb]{0.10,0.09,0.49}{#1}}
\newcommand{\VerbatimStringTok}[1]{\textcolor[rgb]{0.25,0.44,0.63}{#1}}
\newcommand{\WarningTok}[1]{\textcolor[rgb]{0.38,0.63,0.69}{\textbf{\textit{#1}}}}
\setlength{\emergencystretch}{3em} % prevent overfull lines
\providecommand{\tightlist}{%
  \setlength{\itemsep}{0pt}\setlength{\parskip}{0pt}}
\setcounter{secnumdepth}{-\maxdimen} % remove section numbering

\usepackage[a4paper,top=2cm,bottom=1.5cm,left=2cm,right=2cm,marginparwidth=1.5cm]{geometry}
\usepackage{graphicx}
\usepackage{amssymb,amsmath}
\usepackage[french]{babel}
\usepackage[T1]{fontenc}
\usepackage{fancyhdr}
\usepackage{setspace}
\usepackage[font=small, labelfont=bf]{caption}
%\usepackage[colorlinks=true, allcolors=blue]{hyperref}
\usepackage{textcomp}
\usepackage{fancyvrb}
\usepackage{incgraph}
\usepackage{algorithm}
\usepackage{pdfpages}

% Mise en page 
\newcommand{\HRule}[1]{\rule{\linewidth}{#1}}
\onehalfspacing

\pagestyle{fancy}
\fancyhf{}
\setlength\headheight{15pt}
 \setlength{\marginparwidth}{2cm}
\usepackage[
    left = \flqq{},% 
    right = \frqq{},% 
    leftsub = \flq{},% 
    rightsub = \frq{} %
    ]{dirtytalk}
    \setlength\headheight{15pt}
    
    % Ligne de Haut de page (Nom + UV)
    \fancyhead[L]{SAVARY Tobias UTC - GI02} 
    \fancyhead[R]{TODO : UV - Rapport TPX}

    % Bas de page (Nomero de page + UTC)
    \fancyfoot[R]{\thepage}
    \fancyfoot[L]{Université de Technologie de Compiègne}
     \setlength {\marginparwidth}{2cm}
     \renewcommand{\footrulewidth}{.5pt}


\author{}
\date{}

\begin{document}

\hypertarget{td2-exo1}{%
\subsection{TD2 : Exo1}\label{td2-exo1}}

\begin{Shaded}
\begin{Highlighting}[]
\PreprocessorTok{\#include }\ImportTok{\textless{}stdio.h\textgreater{}}
\PreprocessorTok{\#include }\ImportTok{\textless{}signal.h\textgreater{}}
\PreprocessorTok{\#include }\ImportTok{\textless{}unistd.h\textgreater{}}
\PreprocessorTok{\#include }\ImportTok{\textless{}sys/types.h\textgreater{}}
\PreprocessorTok{\#include }\ImportTok{\textless{}stdlib.h\textgreater{}}
\PreprocessorTok{\#include }\ImportTok{\textless{}time.h\textgreater{}}


\KeywordTok{struct}\NormalTok{ sigaction ActFils;}
\KeywordTok{struct}\NormalTok{ sigaction ActPere;}
\NormalTok{pid\_t fils\_pid;}

\DataTypeTok{void}\NormalTok{ fonction\_fils()\{}
    \DataTypeTok{int}\NormalTok{ val = rand() \% }\DecValTok{30}\NormalTok{ + }\DecValTok{10}\NormalTok{;}
\NormalTok{    printf(}\StringTok{"\%d"}\NormalTok{, val);}
\NormalTok{    fflush(stdout);}
\NormalTok{\}}

\DataTypeTok{void}\NormalTok{ fonction\_pere()\{}
\NormalTok{    kill(fils\_pid, SIGUSR1);}
\NormalTok{    alarm(}\DecValTok{5}\NormalTok{);}
\NormalTok{\}}

\DataTypeTok{int}\NormalTok{ main()\{}
\NormalTok{    printf(}\StringTok{"INIT"}\NormalTok{);}
\NormalTok{    fils\_pid = fork();}
    \ControlFlowTok{if}\NormalTok{ (fils\_pid == }\DecValTok{0}\NormalTok{)\{}
        \CommentTok{// printf("Je suis le fils");}
\NormalTok{        ActFils.sa\_handler = fonction\_fils;}
\NormalTok{        sigaction(SIGUSR1, \&ActFils, }\DecValTok{0}\NormalTok{);}
        \ControlFlowTok{while}\NormalTok{(}\DecValTok{1}\NormalTok{)\{}
\NormalTok{            pause();}
\NormalTok{        \}}
\NormalTok{    \} }\ControlFlowTok{else} \ControlFlowTok{if}\NormalTok{ (fils\_pid \textgreater{} }\DecValTok{0}\NormalTok{)\{}
        \CommentTok{// printf("Je suis le pere");}
\NormalTok{        ActPere.sa\_handler = fonction\_pere;}
\NormalTok{        sigaction(SIGALRM, \&ActPere, }\DecValTok{0}\NormalTok{);}
\NormalTok{        alarm(}\DecValTok{5}\NormalTok{);}
        \ControlFlowTok{while}\NormalTok{(}\DecValTok{1}\NormalTok{)\{}
\NormalTok{            sleep(}\DecValTok{1}\NormalTok{);}
\NormalTok{            printf(}\StringTok{"{-}"}\NormalTok{);}
\NormalTok{            fflush(stdout); }\CommentTok{// Pour afficher le printf sans attendre le retour à la ligne (vide le buffer)}
\NormalTok{        \}}
\NormalTok{    \} }\ControlFlowTok{else}\NormalTok{ \{}
\NormalTok{        printf(}\StringTok{"Erreur de création du fils"}\NormalTok{);}
\NormalTok{    \}}

\NormalTok{\}}
\end{Highlighting}
\end{Shaded}

\hypertarget{td2-exo2}{%
\subsection{TD2 : Exo2}\label{td2-exo2}}

\begin{Shaded}
\begin{Highlighting}[]
\PreprocessorTok{\#include }\ImportTok{\textless{}stdio.h\textgreater{}}
\PreprocessorTok{\#include }\ImportTok{\textless{}signal.h\textgreater{}}
\PreprocessorTok{\#include }\ImportTok{\textless{}unistd.h\textgreater{}}
\PreprocessorTok{\#include }\ImportTok{\textless{}sys/types.h\textgreater{}}
\PreprocessorTok{\#include }\ImportTok{\textless{}stdlib.h\textgreater{}}
\PreprocessorTok{\#include }\ImportTok{\textless{}time.h\textgreater{}}


\KeywordTok{struct}\NormalTok{ sigaction ActFils;}
\KeywordTok{struct}\NormalTok{ sigaction ActPere;}
\NormalTok{pid\_t fils\_pid;}
\DataTypeTok{int}\NormalTok{ cpt = }\DecValTok{0}\NormalTok{;}
\DataTypeTok{char}\NormalTok{ cfils = }\CharTok{\textquotesingle{}a\textquotesingle{}}\NormalTok{;}
\DataTypeTok{char}\NormalTok{ cpere = }\CharTok{\textquotesingle{}A\textquotesingle{}}\NormalTok{;}

\DataTypeTok{void}\NormalTok{ fonction\_fils()\{}
\NormalTok{    cpt++;}
    \ControlFlowTok{for}\NormalTok{ (}\DataTypeTok{int}\NormalTok{ i = }\DecValTok{0}\NormalTok{; i \textless{} cpt \&\& cfils \textless{}= }\CharTok{\textquotesingle{}z\textquotesingle{}}\NormalTok{; i++)\{}
\NormalTok{        printf(}\StringTok{"\%c"}\NormalTok{, cfils);}
\NormalTok{        fflush(stdout);}
\NormalTok{        cfils++;}
\NormalTok{    \}}
\NormalTok{    kill(getppid(), SIGUSR1);}
    \ControlFlowTok{if}\NormalTok{(cfils \textgreater{} }\CharTok{\textquotesingle{}z\textquotesingle{}}\NormalTok{)\{}
\NormalTok{        exit(}\DecValTok{0}\NormalTok{);}
\NormalTok{    \}}
\NormalTok{\}}

\DataTypeTok{void}\NormalTok{ fonction\_pere()\{}
\NormalTok{    cpt++;}
    \ControlFlowTok{for}\NormalTok{ (}\DataTypeTok{int}\NormalTok{ i = }\DecValTok{0}\NormalTok{; i \textless{} cpt \&\& cpere \textless{}= }\CharTok{\textquotesingle{}Z\textquotesingle{}}\NormalTok{; i++)\{}
\NormalTok{        printf(}\StringTok{"\%c"}\NormalTok{, cpere);}
\NormalTok{        fflush(stdout);}
\NormalTok{        cpere++;}
\NormalTok{    \}}
    \ControlFlowTok{if}\NormalTok{(cpere \textgreater{} }\CharTok{\textquotesingle{}Z\textquotesingle{}}\NormalTok{)\{}
\NormalTok{        exit(}\DecValTok{0}\NormalTok{);}
\NormalTok{    \}}
\NormalTok{    kill(fils\_pid, SIGUSR1);}
\NormalTok{\}}

\DataTypeTok{int}\NormalTok{ main()\{}
\NormalTok{    fils\_pid = fork();}
    \ControlFlowTok{if}\NormalTok{ (fils\_pid == }\DecValTok{0}\NormalTok{)\{}
\NormalTok{        ActFils.sa\_handler = fonction\_fils;}
\NormalTok{        sigaction(SIGUSR1, \&ActFils, }\DecValTok{0}\NormalTok{);}
        \ControlFlowTok{while}\NormalTok{ (}\DecValTok{1}\NormalTok{)}
\NormalTok{        \{}
\NormalTok{            pause();}
\NormalTok{        \}}
        
\NormalTok{    \} }\ControlFlowTok{else} \ControlFlowTok{if}\NormalTok{ (fils\_pid \textgreater{} }\DecValTok{0}\NormalTok{)\{}
\NormalTok{        ActPere.sa\_handler = fonction\_pere;}
\NormalTok{        sigaction(SIGUSR1, \&ActPere, }\DecValTok{0}\NormalTok{);}
\NormalTok{        sleep(}\DecValTok{1}\NormalTok{);}
\NormalTok{        kill(fils\_pid, SIGUSR1);}
        \ControlFlowTok{while}\NormalTok{ (}\DecValTok{1}\NormalTok{)}
\NormalTok{        \{}
\NormalTok{            pause();}
\NormalTok{        \}}
\NormalTok{    \} }\ControlFlowTok{else}\NormalTok{ \{}
\NormalTok{        printf(}\StringTok{"Erreur de création du fils"}\NormalTok{);}
\NormalTok{    \}}
\NormalTok{    printf(}\StringTok{"}\SpecialCharTok{\textbackslash{}n}\StringTok{"}\NormalTok{);}

\NormalTok{\}}
\end{Highlighting}
\end{Shaded}

\hypertarget{td2-exo3}{%
\subsection{TD2 : Exo3}\label{td2-exo3}}

\begin{Shaded}
\begin{Highlighting}[]
\PreprocessorTok{\#include }\ImportTok{\textless{}signal.h\textgreater{}}
\PreprocessorTok{\#include }\ImportTok{\textless{}unistd.h\textgreater{}}
\PreprocessorTok{\#include }\ImportTok{\textless{}sys/types.h\textgreater{}}
\PreprocessorTok{\#include }\ImportTok{\textless{}stdlib.h\textgreater{}}
\PreprocessorTok{\#include }\ImportTok{\textless{}time.h\textgreater{}}
\PreprocessorTok{\#include }\ImportTok{\textless{}sys/wait.h\textgreater{}}


\CommentTok{//Variables globales}
\KeywordTok{struct}\NormalTok{ sigaction ActFils;}
\KeywordTok{struct}\NormalTok{ sigaction ActPere;}
\NormalTok{pid\_t fils\_pid;}

\CommentTok{//Compteurs de SIGINT}
\CommentTok{//Hypothèse : les compteurs de SIGINT sont indépendants }
\DataTypeTok{int}\NormalTok{ cptPere = }\DecValTok{0}\NormalTok{;}
\DataTypeTok{int}\NormalTok{ cptFils = }\DecValTok{0}\NormalTok{;}


\CommentTok{// Fonction fils}
\DataTypeTok{void}\NormalTok{ captfils()\{}
\NormalTok{    rectvert(}\DecValTok{5}\NormalTok{);}
\NormalTok{    cptFils++;}
    
    \CommentTok{// 3 SIGINT Recu pour le processus fils}
    \ControlFlowTok{if}\NormalTok{(cptFils == }\DecValTok{3}\NormalTok{)\{}
\NormalTok{        detruitrec();}
\NormalTok{        exit(}\DecValTok{0}\NormalTok{);}
\NormalTok{    \}}
\NormalTok{\}}


\CommentTok{//Fonction pere}
\DataTypeTok{void}\NormalTok{ captpere()\{}
\NormalTok{    cptPere++;}
\NormalTok{    printf(}\StringTok{"PERE \%d : signal \%d recu}\SpecialCharTok{\textbackslash{}n}\StringTok{"}\NormalTok{, getpid(),cptPere);}
\NormalTok{    fflush(stdout);}
    
    \CommentTok{//Si 3 SIGINT reçus pour le père}
    \ControlFlowTok{if}\NormalTok{(cptPere == }\DecValTok{3}\NormalTok{)\{}
\NormalTok{        printf(}\StringTok{"PERE : fin du pere , trois signals sont deja recu}\SpecialCharTok{\textbackslash{}n}\StringTok{"}\NormalTok{);}
\NormalTok{        fflush(stdout);}
\NormalTok{        exit(}\DecValTok{0}\NormalTok{);}
\NormalTok{    \}}
\NormalTok{\}}


\CommentTok{//{-}{-}{-}{-}{-}{-}{-}{-}{-}{-}{-}{-}{-}{-}{-}{-}{-}{-}{-}{-}{-}{-}{-}{-}{-}{-}{-}{-}{-}{-}{-}{-}{-}{-}{-}{-}{-}{-}{-}{-}{-}{-}{-}{-}{-}{-}{-}{-}{-}{-}{-}{-}{-}{-}{-}{-}{-}{-}{-}{-}{-}{-}{-}{-}{-}{-}{-}{-}{-}{-}{-}{-}{-}{-}{-}{-}{-}{-}{-}{-}{-}{-}{-}{-}{-}{-}{-}{-}{-}{-}}

\CommentTok{//Main}

\DataTypeTok{int}\NormalTok{ main()\{}
\NormalTok{    fils\_pid = fork();}
    
    \CommentTok{//Partie Processus fils}
    \ControlFlowTok{if}\NormalTok{ (fils\_pid == }\DecValTok{0}\NormalTok{)\{}
    \DataTypeTok{int}\NormalTok{ i =}\DecValTok{0}\NormalTok{;}
\NormalTok{        initrec();}
\NormalTok{        ActFils.sa\_handler = captfils;}
\NormalTok{        sigaction(SIGINT, \&ActFils, }\DecValTok{0}\NormalTok{);}
        
        \CommentTok{//Tant qu\textquotesingle{}on ne clic pas sur le boutton de fin}
        \ControlFlowTok{while}\NormalTok{ (i != {-}}\DecValTok{1}\NormalTok{)}
\NormalTok{        \{}
\NormalTok{            i = attendreclic();}
            
            \CommentTok{//On clic sur 0}
            \ControlFlowTok{if}\NormalTok{(i==}\DecValTok{0}\NormalTok{)\{}
\NormalTok{                printf(}\StringTok{"FILS envoit au PERE}\SpecialCharTok{\textbackslash{}n}\StringTok{FILS :  pid du mon pere est : \%d}\SpecialCharTok{\textbackslash{}n}\StringTok{"}\NormalTok{, getppid());}
\NormalTok{                fflush (stdout);}
            
            \CommentTok{//Envoie du SIGINT au père}
\NormalTok{                kill(getppid(), SIGINT);}
\NormalTok{            \}}
            
            \CommentTok{//On clic sur autre chose que 0}
            \ControlFlowTok{else}\NormalTok{ \{}
\NormalTok{            printf(}\StringTok{"FILS :  pid de mon pere est : \%d }\SpecialCharTok{\textbackslash{}n}\StringTok{"}\NormalTok{, getppid());}
\NormalTok{            fflush(stdout);}
\NormalTok{            \}}
\NormalTok{        \}}
        
        \CommentTok{// Après clic sur fin }
\NormalTok{    printf(}\StringTok{"FILS :fin du fils apres clic sur FIN}\SpecialCharTok{\textbackslash{}n}\StringTok{"}\NormalTok{);}
\NormalTok{    fflush(stdout);}
\NormalTok{    exit(}\DecValTok{0}\NormalTok{);}
\NormalTok{    \} }
    
    \CommentTok{// Partie processus père}
    \ControlFlowTok{else} \ControlFlowTok{if}\NormalTok{ (fils\_pid \textgreater{} }\DecValTok{0}\NormalTok{)\{}
\NormalTok{        ActPere.sa\_handler = captpere;}
\NormalTok{        sigaction(SIGINT, \&ActPere, }\DecValTok{0}\NormalTok{);}
        
        \CommentTok{// Attente pour être sur que le sigaction }
        \CommentTok{// du fils a bien été pris en compte}
\NormalTok{        sleep(}\DecValTok{1}\NormalTok{);}
        \DataTypeTok{int}\NormalTok{ n;}
        \ControlFlowTok{while}\NormalTok{ (}\DecValTok{1}\NormalTok{)}
\NormalTok{        \{}
\NormalTok{            n = sleep(}\DecValTok{10}\NormalTok{);}
\NormalTok{            printf(}\StringTok{"temps restant : \%d }\SpecialCharTok{\textbackslash{}n}\StringTok{"}\NormalTok{, n);}
\NormalTok{            fflush(stdout);}
\NormalTok{        \}}
        
\NormalTok{    \} }
    
    \CommentTok{// Si on a pas réussi à créer le fils}
    \ControlFlowTok{else}\NormalTok{ \{}
\NormalTok{        printf(}\StringTok{"Erreur de création du fils"}\NormalTok{);}
\NormalTok{        fflush(stdout);}
\NormalTok{    \}}
\NormalTok{    printf(}\StringTok{"}\SpecialCharTok{\textbackslash{}n}\StringTok{"}\NormalTok{);}
\NormalTok{    fflush(stdout);}
    \ControlFlowTok{return} \DecValTok{0}\NormalTok{;}
\NormalTok{\}}
\end{Highlighting}
\end{Shaded}

\hypertarget{td4-exo1}{%
\subsection{TD4 : Exo1}\label{td4-exo1}}

\begin{Shaded}
\begin{Highlighting}[]
\PreprocessorTok{\#include }\ImportTok{"sharemem.h"}

\PreprocessorTok{\#define BLKSIZE 1024}

\DataTypeTok{int}\NormalTok{ copierfichier(}\DataTypeTok{int}\NormalTok{ f1, }\DataTypeTok{int}\NormalTok{ f2)\{}
    \DataTypeTok{char}\NormalTok{ buf[BLKSIZE];}
    \DataTypeTok{int}\NormalTok{ octets\_lus, octets\_ecrits, total = }\DecValTok{0}\NormalTok{;}
    \ControlFlowTok{for}\NormalTok{(;;)\{}
        \ControlFlowTok{if}\NormalTok{((octets\_lus = read(f1, buf, BLKSIZE)) \textless{}= }\DecValTok{0}\NormalTok{)\{}
            \ControlFlowTok{break}\NormalTok{;}
\NormalTok{        \}}
        \ControlFlowTok{if}\NormalTok{((octets\_ecrits = write(f2, buf, octets\_lus)) == {-}}\DecValTok{1}\NormalTok{)\{}
            \ControlFlowTok{break}\NormalTok{;}
\NormalTok{        \}}
\NormalTok{        total += octets\_ecrits;}
\NormalTok{    \}}
    \ControlFlowTok{return}\NormalTok{ total;}
\NormalTok{\}}
\end{Highlighting}
\end{Shaded}

\hypertarget{td4-exo2}{%
\subsection{TD4 : Exo2}\label{td4-exo2}}

\begin{Shaded}
\begin{Highlighting}[]
\CommentTok{// surv}
\PreprocessorTok{\#include }\ImportTok{"sharemem.h"}

\PreprocessorTok{\#define BLKSIZE 1024}

\DataTypeTok{int}\NormalTok{ copierfichier(}\DataTypeTok{int}\NormalTok{ f1, }\DataTypeTok{int}\NormalTok{ f2)\{}
    \DataTypeTok{char}\NormalTok{ buf[BLKSIZE];}
    \DataTypeTok{int}\NormalTok{ octets\_lus, octets\_ecrits, total = }\DecValTok{0}\NormalTok{;}
    \ControlFlowTok{for}\NormalTok{(;;)\{}
        \ControlFlowTok{if}\NormalTok{((octets\_lus = read(f1, buf, BLKSIZE)) \textless{}= }\DecValTok{0}\NormalTok{)\{}
            \ControlFlowTok{break}\NormalTok{;}
\NormalTok{        \}}
        \ControlFlowTok{if}\NormalTok{((octets\_ecrits = write(f2, buf, octets\_lus)) == {-}}\DecValTok{1}\NormalTok{)\{}
            \ControlFlowTok{break}\NormalTok{;}
\NormalTok{        \}}
\NormalTok{        total += octets\_ecrits;}
\NormalTok{    \}}
    \ControlFlowTok{return}\NormalTok{ total;}
\NormalTok{\}}

\DataTypeTok{int}\NormalTok{ main(}\DataTypeTok{int}\NormalTok{ argc, }\DataTypeTok{char}\NormalTok{ *argv[])\{}
    \DataTypeTok{int}\NormalTok{ octets\_lus, childpid, fd, fd1, fd2;}
    \ControlFlowTok{if}\NormalTok{(argc == }\DecValTok{3}\NormalTok{)\{}
\NormalTok{        fprintf(stderr,}\StringTok{"Usage: \%s \textless{}fichier1\textgreater{} \textless{}fichier2\textgreater{} }\SpecialCharTok{\textbackslash{}n}\StringTok{"}\NormalTok{, argv[}\DecValTok{0}\NormalTok{]);}
        \ControlFlowTok{return} \DecValTok{1}\NormalTok{;}
\NormalTok{    \}}
    \ControlFlowTok{if}\NormalTok{(((fd1 = open(argv[}\DecValTok{1}\NormalTok{], O\_RDONLY)) == {-}}\DecValTok{1}\NormalTok{) || (fd2 = open(argv[}\DecValTok{2}\NormalTok{], O\_RDONLY) == {-}}\DecValTok{1}\NormalTok{))\{ }\CommentTok{// Vérifie si c\textquotesingle{}est bien ouvert}
\NormalTok{        perror(}\StringTok{"Echec"}\NormalTok{);}
        \ControlFlowTok{return} \DecValTok{1}\NormalTok{;}
\NormalTok{    \}}
    \ControlFlowTok{if}\NormalTok{((childpid = fork()) == {-}}\DecValTok{1}\NormalTok{)\{}
\NormalTok{        perror(}\StringTok{"Echec"}\NormalTok{);}
        \ControlFlowTok{return} \DecValTok{1}\NormalTok{;}
\NormalTok{    \}}
    \ControlFlowTok{if}\NormalTok{(childpid\textgreater{}}\DecValTok{0}\NormalTok{) }\CommentTok{// Parent code}
\NormalTok{        fd = fd1;}
    \ControlFlowTok{else}  \CommentTok{// Child code}
\NormalTok{        fd = fd2;}
\NormalTok{    octets\_lus = copierfichier(fd, STDOUT\_FILENO);}
\NormalTok{    fprintf(stderr, }\StringTok{"Octets Lus : \%d }\SpecialCharTok{\textbackslash{}n}\StringTok{"}\NormalTok{, octets\_lus);}
    \ControlFlowTok{return} \DecValTok{0}\NormalTok{;}
\NormalTok{\}}


\CommentTok{// Compile : gcc {-}u prgm surv.c}
\CommentTok{// Run : ./prgm f1.data fé.data}
\CommentTok{// SURVSHM}
\PreprocessorTok{\#include }\ImportTok{"sharemem.h"}

\PreprocessorTok{\#define BLKSIZE 1024}

\PreprocessorTok{\#define PERM (S\_IRUSR | S\_IWUSR)}
\PreprocessorTok{\#define SHMZ 27}

\DataTypeTok{int}\NormalTok{ copierfichier(}\DataTypeTok{int}\NormalTok{ f1, }\DataTypeTok{int}\NormalTok{ f2)\{}
    \DataTypeTok{char}\NormalTok{ buf[BLKSIZE];}
    \DataTypeTok{int}\NormalTok{ octets\_lus, octets\_ecrits, total = }\DecValTok{0}\NormalTok{;}
    \ControlFlowTok{for}\NormalTok{(;;)\{}
        \ControlFlowTok{if}\NormalTok{((octets\_lus = read(f1, buf, BLKSIZE)) \textless{}= }\DecValTok{0}\NormalTok{)\{}
            \ControlFlowTok{break}\NormalTok{;}
\NormalTok{        \}}
        \ControlFlowTok{if}\NormalTok{((octets\_ecrits = write(f2, buf, octets\_lus)) == {-}}\DecValTok{1}\NormalTok{)\{}
            \ControlFlowTok{break}\NormalTok{;}
\NormalTok{        \}}
\NormalTok{        total += octets\_ecrits;}
\NormalTok{    \}}
    \ControlFlowTok{return}\NormalTok{ total;}
\NormalTok{\}}

\DataTypeTok{int}\NormalTok{ statchAndRemove(}\DataTypeTok{int}\NormalTok{ shmid, }\DataTypeTok{char}\NormalTok{ *shmdt)\{}
    \DataTypeTok{int}\NormalTok{ error = }\DecValTok{0}\NormalTok{;}
    \ControlFlowTok{if}\NormalTok{(shmdt(shmdt) == {-}}\DecValTok{1}\NormalTok{)\{}
\NormalTok{        error = errno;}
\NormalTok{    \}}
    \ControlFlowTok{if}\NormalTok{(shmdt(shmid, IPC\_RMID, NULL) == {-}}\DecValTok{1}\NormalTok{ \&\& !error)\{}
\NormalTok{        error = errno;}
\NormalTok{    \}}
    \ControlFlowTok{if}\NormalTok{(!error)\{}
        \ControlFlowTok{return} \DecValTok{0}\NormalTok{;}
\NormalTok{    \}}
\NormalTok{\}}
\end{Highlighting}
\end{Shaded}

\hypertarget{td4-exo3}{%
\subsection{TD4 : Exo3}\label{td4-exo3}}

\begin{Shaded}
\begin{Highlighting}[]
\CommentTok{// INIFIC}
\PreprocessorTok{\#include }\ImportTok{\textless{}stdio.h\textgreater{}}
\PreprocessorTok{\#include }\ImportTok{\textless{}fcntl.h\textgreater{}}

\DataTypeTok{void}\NormalTok{ main()\{}
    \DataTypeTok{int}\NormalTok{ tab1[}\DecValTok{10}\NormalTok{]=\{}\DecValTok{11}\NormalTok{,}\DecValTok{22}\NormalTok{,}\DecValTok{33}\NormalTok{,}\DecValTok{44}\NormalTok{,}\DecValTok{55}\NormalTok{,}\DecValTok{66}\NormalTok{,}\DecValTok{77}\NormalTok{,}\DecValTok{88}\NormalTok{,}\DecValTok{99}\NormalTok{,}\DecValTok{1000}\NormalTok{\};}
    \DataTypeTok{int}\NormalTok{ fd = open(}\StringTok{"titi.dat"}\NormalTok{, O\_RDWR|O\_CREAT|O\_TRUNC, }\BaseNTok{0666}\NormalTok{);}
    \ControlFlowTok{if}\NormalTok{(fd == {-}}\DecValTok{1}\NormalTok{)\{}
\NormalTok{        printf(}\StringTok{"Erreur lors de l\textquotesingle{}ouverture du fichier}\SpecialCharTok{\textbackslash{}n}\StringTok{"}\NormalTok{);}
\NormalTok{        perror(}\StringTok{"open"}\NormalTok{);}
\NormalTok{        exit({-}}\DecValTok{1}\NormalTok{);}
\NormalTok{    \}}
\NormalTok{    write(fd, tab1, }\DecValTok{10}\NormalTok{*}\KeywordTok{sizeof}\NormalTok{(}\DataTypeTok{int}\NormalTok{));}
\NormalTok{    close(fd);}
\NormalTok{    printf(}\StringTok{"Fichier initialisé}\SpecialCharTok{\textbackslash{}n}\StringTok{"}\NormalTok{);}
\NormalTok{\}}
\CommentTok{// LIREFIC}
\DataTypeTok{void}\NormalTok{ main()\{}
    \DataTypeTok{int}\NormalTok{ tab2[}\DecValTok{10}\NormalTok{];}
    \DataTypeTok{int}\NormalTok{ i;}
    \CommentTok{// Ouverture du fichier}
    \DataTypeTok{int}\NormalTok{ fd = open(}\StringTok{"titi.dat"}\NormalTok{, O\_RDWR, }\BaseNTok{0666}\NormalTok{);}
    \ControlFlowTok{if}\NormalTok{(fd == {-}}\DecValTok{1}\NormalTok{)\{}
\NormalTok{        printf(}\StringTok{"Erreur lors de l\textquotesingle{}ouverture du fichier}\SpecialCharTok{\textbackslash{}n}\StringTok{"}\NormalTok{);}
\NormalTok{        perror(}\StringTok{"open"}\NormalTok{);}
\NormalTok{        exit({-}}\DecValTok{1}\NormalTok{);}
\NormalTok{    \}}
    \CommentTok{//Lecture du fichier}
\NormalTok{    read(fd, tab2, }\DecValTok{10}\NormalTok{*}\KeywordTok{sizeof}\NormalTok{(}\DataTypeTok{int}\NormalTok{));}
\NormalTok{    close(fd);}
    \ControlFlowTok{for}\NormalTok{(i=}\DecValTok{0}\NormalTok{; i\textless{}}\DecValTok{10}\NormalTok{; i++) }
\NormalTok{        printf(}\StringTok{"\%d}\SpecialCharTok{\textbackslash{}n}\StringTok{"}\NormalTok{,tab2[i]);}
\NormalTok{\}}

\CommentTok{// MODIFIC}
\PreprocessorTok{\#include }\ImportTok{\textless{}stdio.h\textgreater{}}
\PreprocessorTok{\#include }\ImportTok{\textless{}stdlib.h\textgreater{}}
\PreprocessorTok{\#include }\ImportTok{\textless{}fcntl.h\textgreater{}}
\PreprocessorTok{\#include }\ImportTok{\textless{}unistd.h\textgreater{}}
\PreprocessorTok{\#include }\ImportTok{\textless{}sys/types.h\textgreater{}}
\PreprocessorTok{\#include }\ImportTok{\textless{}sys/stat.h\textgreater{}}
\PreprocessorTok{\#include }\ImportTok{\textless{}sys/mman.h\textgreater{}}


\DataTypeTok{void}\NormalTok{ main(}\DataTypeTok{int}\NormalTok{ argc, }\DataTypeTok{char}\NormalTok{* argv[])\{}
    \DataTypeTok{int}\NormalTok{ i;}
    \KeywordTok{struct}\NormalTok{ stat file\_stat;}
    \DataTypeTok{char}\NormalTok{* addr;}
    \DataTypeTok{int}\NormalTok{ file;}
    \DataTypeTok{int}\NormalTok{* tab1;}
     \CommentTok{// Ouverture du fichier}
\NormalTok{    file = open(}\StringTok{"titi.dat"}\NormalTok{,O\_RDWR, }\BaseNTok{0666}\NormalTok{);}
    \ControlFlowTok{if}\NormalTok{(file == {-}}\DecValTok{1}\NormalTok{)\{}
\NormalTok{        printf(}\StringTok{"Erreur lors de l\textquotesingle{}ouverture du fichier}\SpecialCharTok{\textbackslash{}n}\StringTok{"}\NormalTok{);}
\NormalTok{        perror(}\StringTok{"open"}\NormalTok{);}
\NormalTok{        exit({-}}\DecValTok{1}\NormalTok{);}
\NormalTok{    \}}
    
    \CommentTok{// Récupération des informations du fichier}
\NormalTok{    fstat(file, \&file\_stat);}

    \CommentTok{//Ouverture de la mémoire partagé}
\NormalTok{    tab1 = mmap(NULL,file\_stat.st\_size,PROT\_READ | PROT\_WRITE,MAP\_SHARED,file,}\DecValTok{0}\NormalTok{);}
\NormalTok{    close(file);}
    \CommentTok{//Boucle sur i}
    \ControlFlowTok{while}\NormalTok{ (}\DecValTok{1}\NormalTok{)\{}
\NormalTok{        printf(}\StringTok{"Rentrez i }\SpecialCharTok{\textbackslash{}n}\StringTok{"}\NormalTok{);}
\NormalTok{        scanf(}\StringTok{" \%d"}\NormalTok{,\&i);}
        \ControlFlowTok{if}\NormalTok{ (i == }\DecValTok{99}\NormalTok{)\{}
            \CommentTok{// Sortie du programme}
\NormalTok{            munmap(tab1, file\_stat.st\_size);}
\NormalTok{            exit(}\DecValTok{0}\NormalTok{);}
\NormalTok{        \}}
        \CommentTok{// Incrémentation du nombre de 1 à l\textquotesingle{}indice [i] dans le segment de la mémoire partagée}
        \ControlFlowTok{else} \ControlFlowTok{if}\NormalTok{( i \textgreater{}= }\DecValTok{0}\NormalTok{ \&\& i \textless{}= }\DecValTok{9}\NormalTok{)\{}
\NormalTok{            tab1[i]++;}
\NormalTok{        \}}
\NormalTok{    \}}

    \CommentTok{// Libération de la mémoire partagée}
\NormalTok{    munmap(tab1, file\_stat.st\_size);}
    \ControlFlowTok{return} \DecValTok{0}\NormalTok{;}
\NormalTok{\}}

\CommentTok{// SHOWFIC}
\PreprocessorTok{\#include }\ImportTok{\textless{}stdio.h\textgreater{}}
\PreprocessorTok{\#include }\ImportTok{\textless{}stdlib.h\textgreater{}}
\PreprocessorTok{\#include }\ImportTok{\textless{}fcntl.h\textgreater{}}
\PreprocessorTok{\#include }\ImportTok{\textless{}unistd.h\textgreater{}}
\PreprocessorTok{\#include }\ImportTok{\textless{}sys/types.h\textgreater{}}
\PreprocessorTok{\#include }\ImportTok{\textless{}sys/stat.h\textgreater{}}
\PreprocessorTok{\#include }\ImportTok{\textless{}sys/mman.h\textgreater{}}


\DataTypeTok{void}\NormalTok{ main(}\DataTypeTok{int}\NormalTok{ argc, }\DataTypeTok{char}\NormalTok{* argv[])\{}
    \DataTypeTok{int}\NormalTok{ i;}
    \KeywordTok{struct}\NormalTok{ stat file\_stat;}
    \DataTypeTok{char}\NormalTok{* addr;}
    \DataTypeTok{int}\NormalTok{ file;}
    \DataTypeTok{int}\NormalTok{* tab1;}

    \CommentTok{// Ouverture du fichier}
\NormalTok{    file = open(}\StringTok{"titi.dat"}\NormalTok{,O\_RDWR, }\BaseNTok{0666}\NormalTok{);}
    \ControlFlowTok{if}\NormalTok{(file == {-}}\DecValTok{1}\NormalTok{)\{}
\NormalTok{        printf(}\StringTok{"Erreur lors de l\textquotesingle{}ouverture du fichier}\SpecialCharTok{\textbackslash{}n}\StringTok{"}\NormalTok{);}
\NormalTok{        perror(}\StringTok{"open"}\NormalTok{);}
\NormalTok{        exit({-}}\DecValTok{1}\NormalTok{);}
\NormalTok{    \}}
    
    \CommentTok{// Récupération des informations du fichier}
\NormalTok{    fstat(file, \&file\_stat);}

    \CommentTok{//Ouverture de la mémoire partagé}
\NormalTok{    tab1 = mmap(NULL,file\_stat.st\_size,PROT\_READ | PROT\_WRITE,MAP\_SHARED,file,}\DecValTok{0}\NormalTok{);}
\NormalTok{    close(file);}
    \CommentTok{//Boucle sur i}
    \ControlFlowTok{while}\NormalTok{ (}\DecValTok{1}\NormalTok{)\{}
\NormalTok{        printf(}\StringTok{"Rentrez i }\SpecialCharTok{\textbackslash{}n}\StringTok{"}\NormalTok{);}
\NormalTok{        scanf(}\StringTok{" \%d"}\NormalTok{,\&i);}
        \ControlFlowTok{if}\NormalTok{ (i == }\DecValTok{99}\NormalTok{)\{}
            \CommentTok{// Sortie du programme}
\NormalTok{            munmap(tab1, file\_stat.st\_size);}
\NormalTok{            exit(}\DecValTok{0}\NormalTok{);}
\NormalTok{        \}}
        \CommentTok{// Affichage du contenu de la mémoire partagée}
        \ControlFlowTok{else} \ControlFlowTok{if}\NormalTok{( i \textgreater{}= }\DecValTok{0}\NormalTok{ \&\& i \textless{}= }\DecValTok{9}\NormalTok{)\{}
            \ControlFlowTok{for}\NormalTok{ (}\DataTypeTok{int}\NormalTok{ ctp = }\DecValTok{0}\NormalTok{ ; ctp \textless{}= }\DecValTok{9}\NormalTok{ ; ctp++)}
\NormalTok{                printf(}\StringTok{"\%d}\SpecialCharTok{\textbackslash{}n}\StringTok{"}\NormalTok{, tab1[ctp]);}
\NormalTok{        \}}
\NormalTok{    \}}
    
    \CommentTok{// Libération de la mémoire partagée}
\NormalTok{    munmap(tab1, file\_stat.st\_size);}
    \ControlFlowTok{return} \DecValTok{0}\NormalTok{;}
\NormalTok{\}}
\end{Highlighting}
\end{Shaded}

\hypertarget{td4-exo4}{%
\subsection{TD4 : Exo4}\label{td4-exo4}}

\begin{Shaded}
\begin{Highlighting}[]
\PreprocessorTok{\#include }\ImportTok{\textless{}sys/mman.h\textgreater{}}
\PreprocessorTok{\#include }\ImportTok{\textless{}sys/stat.h\textgreater{}}
\PreprocessorTok{\#include }\ImportTok{\textless{}fcntl.h\textgreater{}}
\PreprocessorTok{\#include }\ImportTok{\textless{}unistd.h\textgreater{}}
\PreprocessorTok{\#include }\ImportTok{\textless{}stdio.h\textgreater{}}
\PreprocessorTok{\#include }\ImportTok{\textless{}stdlib.h\textgreater{}}

\DataTypeTok{int}\NormalTok{ main(}\DataTypeTok{int}\NormalTok{ argc, }\DataTypeTok{char}\NormalTok{* argv[]) \{}
    \ControlFlowTok{if}\NormalTok{(argc != }\DecValTok{2}\NormalTok{) \{}
\NormalTok{        printf(}\StringTok{"Syntaxe: \%s fichier\_à\_inverser}\SpecialCharTok{\textbackslash{}n}\StringTok{"}\NormalTok{, argv[}\DecValTok{0}\NormalTok{]);}
\NormalTok{        exit({-}}\DecValTok{1}\NormalTok{);}
\NormalTok{    \}}

    \DataTypeTok{int}\NormalTok{ fd = open(argv[}\DecValTok{1}\NormalTok{], O\_RDWR);}
    \ControlFlowTok{if}\NormalTok{(fd == {-}}\DecValTok{1}\NormalTok{) \{}
\NormalTok{        printf(}\StringTok{"Erreur lors de l\textquotesingle{}ouverture du fichier}\SpecialCharTok{\textbackslash{}n}\StringTok{"}\NormalTok{);}
\NormalTok{        perror(}\StringTok{"open"}\NormalTok{);}
\NormalTok{        exit({-}}\DecValTok{1}\NormalTok{);}
\NormalTok{    \}}

    \KeywordTok{struct}\NormalTok{ stat sb;}
\NormalTok{    fstat(fd, \&sb); }\CommentTok{// Cette fonction nous permet de récupérer les informations sur le fichier}
\CommentTok{(dans notre cas la taille)}
    \DataTypeTok{int}\NormalTok{ *fichier = mmap(NULL, sb.st\_size, PROT\_READ | PROT\_WRITE, MAP\_SHARED, fd, }\DecValTok{0}\NormalTok{);}

    \DataTypeTok{int}\NormalTok{ *fichier\_inverse = malloc(sb.st\_size); }\CommentTok{// nouvelle zone de mémoire pr le contenu inversé}
    \DataTypeTok{int}\NormalTok{ i = sb.st\_size / }\KeywordTok{sizeof}\NormalTok{(}\DataTypeTok{int}\NormalTok{) {-} }\DecValTok{1}\NormalTok{; }\CommentTok{// On commence à la fin du premier fichier}
    \DataTypeTok{int}\NormalTok{ j = }\DecValTok{0}\NormalTok{; }\CommentTok{// On écrit au début du fichier à inverser}
    \ControlFlowTok{while}\NormalTok{ (i \textgreater{}= }\DecValTok{0}\NormalTok{) \{}
\NormalTok{        fichier\_inverse[j] = fichier[i]; }\CommentTok{// copie des caractères inversés}
\NormalTok{        i{-}{-};}
\NormalTok{        j++;}
\NormalTok{    \}}
\NormalTok{    memcpy(fichier, fichier\_inverse, sb.st\_size); }\CommentTok{// copie de la nouvelle zone de mémoire dans le fichier}
\NormalTok{    printf(}\StringTok{"Le contenu du fichier a été inversé avec succès}\SpecialCharTok{\textbackslash{}n}\StringTok{"}\NormalTok{);}
\NormalTok{    munmap(fichier, sb.st\_size); }\CommentTok{// libération de la mémoire}
\NormalTok{    free(fichier\_inverse); }\CommentTok{// libération de la mémoire alouée dynamiquement}
\NormalTok{    close(fd);}
    \ControlFlowTok{return} \DecValTok{0}\NormalTok{;}
\NormalTok{\}}
\end{Highlighting}
\end{Shaded}


\end{document}
